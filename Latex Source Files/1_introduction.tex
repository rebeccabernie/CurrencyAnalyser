\chapter{Introduction}\label{intro}

Throughout our first three years of Software Development at Galway-Mayo Institute of Technology, we have continuously been encouraged to maintain a comprehensive knowledge of the trends within the technology industry, and to embrace its ever-changing nature.

When initial meetings to discuss possible project ideas began in September 2017, there was a mutual agreement within the team that a concept that would be interesting, distinct from other projects, and most importantly be beneficial and of use in its field, should be pursued. After numerous ideas were considered and after much deliberation, it was decided that a focus on the area of cryptocurrency and more specifically, analysing changes in the market and attempting to decipher trends in prices, would be appropriate. This was deemed a good decision as there was no awareness of any similar projects from previous years, and most importantly, all team members had a keen interest in the topic outside of academia.

A mere five years ago, cryptocurrency was a relatively unheard of phrase to the average individual. In the years since, the likes of Bitcoin and Ethereum have become almost household terms, with many more people investing in various cryptocurrencies and following their repeating rise and decline. While cryptocurrencies were initially a mystery to the average individual, the arrival of user-friendly trading sites has meant they have now become an almost common asset, seen regularly in the news and no doubt discussed over many water coolers. 

Although cryptocurrency is no longer seen as an unobtainable investment, meant only for those with an in-depth knowledge of how to keep their digital currency stored safely and properly, there still exists a mystery surrounding when the best time is to buy or sell. This uncertainty, coupled with personal interest in the field, was the inspiration behind this final year project; price predictions of one of the more popular cryptocurrencies would be calculated and delivered to users in a simple manner that could be understood by anyone with even a basic knowledge of the area of cryptocurrency.

This dissertation aims to first give the reader a good understanding of what exactly cryptocurrency is, how it works and the various technologies behind it. The volatility of cryptocurrency as an asset will be analysed, including the influencing factors in the changing of its prices. Following the initial theoretical chapters we will move to discussing the applied aspect of this project, where the development process and reasoning behind technologies used, among other relevant topics, will be outlined and explained. Finally, the dissertation will conclude with a summary of the project as a whole, along with any discoveries gained throughout the project.

\section{Project Objectives}\label{objectives}

As mentioned previously, the main objective for this project was to make the area of cryptocurrency more accessible to an individual with little knowledge of the field. 

As this project is divided into a research-based dissertation and an applied project, goals will be discussed in relation to each aspect. The objectives for this dissertation are as follows:

\begin{itemize}
    \item \textit{Introduce the concept of this project}: We will provide the reader with an introduction to the project, detailing its inspiration and goals.
    \item\textit{Provide the reader with a rounded understanding of cryptocurrencies}: We will examine where cryptocurrency began, and the concept of modern cryptocurrency in simple terms including how to begin trading cryptocurrency and the underlying technologies. We will then analyse the overall viability of cryptocurrency as an asset, based on its fundamental components.
    \item\textit{Explain to the reader how volatile cryptocurrency prices can be}: Having provided the reader with an understanding of cryptocurrency, we will proceed by discussing the prices of cryptocurrency; how they are determined, and what can inadvertently affect them. We will consider the prediction of prices, or rather the inability to predict prices, examining any known indicators which aid in the uncertain forecasting of prices.
    \item\textit{Describe in detail the applied aspect of this project}: We will examine the approach of the team to the applied project, including methodologies and technologies used, and design and evaluation of the system. Any issues encountered throughout the development process will also be discussed, including how such issues were resolved and what could be done differently in future.
\end{itemize}

With regards to the applied component of this project, our objectives are as follows:

\begin{itemize}
    \item \textit{Create a simple web application which is easy to use and clear to understand}: While the web application is intended to be an extension of this dissertation, it will be developed with even the most inexperienced of users in mind. Any quick internet search indicates most popular websites related to cryptocurrency are daunting at first sight due to the extensive numbers of graphs, percentages, and unfamiliar terminology. One of the most important goals for this application is that it be more encouraging to unfamiliar users than existing complicated sites.
    \item\textit{Deliver cryptocurrency prices to the user}: The web application should bring up-to-date prices for the Bitcoin currency to the user in the form of an easily interpreted graph. The user will be able to view the price in comparison to a traditional currency, such as the Euro.
    \item\textit{Provide an educated guess as to future changes in prices}: Machine Learning and Neural Networks will be integrated to provide the user with an educated estimate as to what a price will change to. It should be noted that the aim is not to predict prices perfectly, as this is impossible due to the variety of factors that influence prices. However, previous price data will be used to attempt to decipher any trend and subsequently produce a price estimate, which will be relayed to the user through the web application. Natural Language Processing techniques could also be implemented to gather and analyse data on user discussions on relevant topics from sources such as Twitter and news websites, as these discussions are proven to have bearing on fluctuations in prices \cite{socmedimpact}.
    \item\textit{Work closely with the given learning outcomes for this project}: All requirements for the research and development process of this project will be strived towards by the team. This includes carrying out extensive research, applying appropriate methodologies and project management techniques, taking advantage of relevant new technologies, and critically evaluating the work including identifying any strengths, weaknesses and future recommendations.
    \item\textit{Conduct work as a team, in a professional manner akin to what is expected in industry}: It is highly important that each member aspires to work with other members as a team, free from disrespect or inequality. No one member should feel as though they should take over the project, and any issues should be resolved in a calm and coordinated manner. It is often observed that friendships are compromised when an important project is added to the equation, but the team aims to keep personal and academic lives as separate as possible for the duration of this project. In the event of a disagreement, the team will endeavour to not let it negatively affect any friendships.
\end{itemize}

\subsection{Metrics for Success or Failure}\label{metrics}
The metrics for success or failure of this project undoubtedly relate closely to the aforementioned objectives. A definitive list of metrics for success for the project as a whole, including dissertation and web application, is as follows:

\begin{itemize}
    \item\textit{An easily understood, cohesive dissertation which can be read from beginning to end by anyone unfamiliar with the topic and leave them with a solid understanding of the ideas discussed.} To measure this, we will ask various friends and family who know little about the area of cryptocurrency to read given sections of the dissertation, and ask for their feedback.
    \item\textit{A simple, effective web application}: Again, to measure this we will ask some friends or family to use the web application for a short time and to give us their opinion of its usability and how informative it was afterwards.
    \item\textit{Educated guesses of future cryptocurrency prices}: We will measure the accuracy of our predictions against the actual data. We will carry out this examination the week prior to submission.
    \item\textit{Teamwork}: We will measure the success of our teamwork by reflecting on how we resolved any issues, and how we conducted ourselves in stressful times. As mentioned in our objectives, we aim to not let any disagreements come between the friendships we had when beginning this project - intact friendships at the conclusion of this project will also be a measure for success or failure.
\end{itemize}
 
\section{Description of Each Chapter}\label{chdescriptions}
In this section, we will briefly outline what each chapter of this dissertation centres around. 

\subsection{Understanding Cryptocurrency}
In chapter \ref{understandingcryptocurrencych}, \textit{Understanding Cryptocurrency}, we first detail where cryptocurrency began, before explaining in simple terms what exactly a cryptocurrency is, and how it differs in various ways from a traditional currency. We then examine the most popular and well-known cryptocurrency, Bitcoin, before delving into some of the technologies behind cryptocurrency, such as blockchain technology. This chapter centres around the viability of cryptocurrency as a whole. 

\subsection{Predicting the Prices of Cryptocurrency}
Chapter \ref{predictingpricesch}, \textit{Predicting the Prices of Cryptocurrency} contrasts the previous chapter by detailing how volatile cryptocurrency can be. We will explain what directly and indirectly affects the prices of cryptocurrency, specifically referencing Bitcoin. We then discuss the Bitcoin "Bubble" and the inability to absolutely predict prices of any cryptocurrency, followed by a more upbeat outlook of making educated estimates of price changes and thus introducing our Currency Analyser web application.

\subsection{Currency Analyser Web Application}
This chapter centres on the applied aspect of this project, building on information discussed in the previous chapters. We examine the methodologies and planning used throughout the project, also dealing with any aspects which could have been planned or managed better. This chapter also explains in detail the technologies used for the applied project, their reasons for being chosen, and any problems that occurred related to the technologies. We then discuss the design of our system including reasoning, followed finally by an overall evaluation of the system. 

\subsection{Conclusion}
The concluding chapter of this dissertation will summarise our initial goals and objectives, reflecting on the theoretical and applied aspects of this project, both conceptually and in practice. We will highlight any findings and any relevant, tangential or even unrelated insights gained during the project life cycle. Finally, we will finish on a positive note with a brief discussion of the team's experience of the project.