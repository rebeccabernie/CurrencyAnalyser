\section{Underlying Technologies}\label{sectechnologies}

\subsection{Python 3}
Python is a high-level, easy to read programming language, ideal for programmers with any level of skill. Python has become one of the most widely used programming languages, certainly due mostly to its simple yet powerful design. Unlike languages such as \textit{Java} and \textit{C}, Python syntax is relatively uncomplicated and thus not as daunting as more traditional languages. This theme carries across into Python documentation \cite{pythondocs}, which remains simple and arguably easier to understand than, for example, Java documentation. 

For the purposes of this project, the Anaconda distribution \cite{anahome} of Python 3 was used, as it is aimed at data science and machine learning applications. Anaconda's virtual environment manager, Anaconda Navigator, is included with the open source distribution and makes installation and updating of packages a much simpler process. There are a large number of external libraries available in the Anaconda distribution of Python 3, including some powerful data science packages like Scikit-learn, TensorFlow, and SciPy. It was for these reasons, as well as past experience with Python 3, that the team chose to work with this particular distribution.

\subsection{The Flask Microframework}
The Flask microframework \cite{flaskhome} is designed for building simple web applications with a Python back-end. By using the Anaconda distribution, Flask can be installed simply through Anaconda Navigator, and all Flask applications can be run on any computer and accessed via opening a browser at \url{http://127.0.0.1:5000/}. The back-end of a Flask application looks very similar to a standard Python file, with added \mintinline{python}{@app.route()} decorators to handle what happens at specific URLs. For example, \mintinline{python}{@app.route("/")} defines the home page of any web application, which in most cases simply loads the specified html file when the URL is called in the front-end. In short, the Flask microframework allows users to create web applications with a Python back-end quickly and easily.

\subsection{Redis}
While MongoDB had initially been intended for use as the store for currency data, research into Redis was carried out based on advice from the team supervisor. Redis, meaning \textit{REmote DIctionary Server}, is an open source in-memory data store which supports many different data types \cite{redishome}. The use of in-memory storage as opposed to disk storage offers a number of advantages, mainly making retrieval of data in Redis extremely fast, capable of performing roughly 110,000 \mintinline{MySQL}{SET} operations or 81,000 \mintinline{MySQL}{GET} operations per second \cite{redistut}. While Redis can be used for multiple purposes, such as caching or message queues, its main purpose in this application was the storage of short-lived currency data.

\subsubsection{Forex-Python}
The Redis store of currency data within this project is updated using a Python script, which obtains the data using the \textcolor{NavyBlue}{\href{https://pypi.python.org/pypi/forex-python}{forex-python}} library \cite{forexpy}. This library contains prices for most traditional currencies, as well as prices for Bitcoin obtained through CoinDesk's \textcolor{NavyBlue}{\href{https://www.coindesk.com/api/}{Bitcoin Price Index API}} \cite{bpiapi}.

\subsection{MongoDB}
MongoDB is an open-source document-based database \cite{mdbdocs}. As mentioned in the above section, MongoDB was initially considered for the storage of currency data, subsequently replaced with Redis. However, the machine learning data and predicted prices of Bitcoin also needed to be stored somewhere. While Redis is suitable for short-lived currency data, it would not be suitable for storing any data related to the machine learning element of this project. It was decided by the team that due to previous knowledge of MongoDB, it would make sense to use it in this respect.

Within a separate Python script, data is obtained from the \textcolor{NavyBlue}{\href{https://pypi.python.org/pypi/forex-python}{forex-python}} library and saved into a MongoDB document. As the machine learning model does not need to be retrained regularly, this is done at a much lesser frequency than that of the Redis script.

\subsection{Vue.js}
Vue.js is a JavaScript framework for building user interfaces, designed to be capable of integration into a project at any stage. Despite the team having no experience with Vue.js, the technology was adopted following advice from the team's supervisor and some intensive research and tutorial-based learning. While Vue.js was initially difficult to adapt to, the wide variety of tutorials and documentation available online greatly helped. The team did briefly consider changing to an alternative framework, but determined it was best to continue with Vue.js due to the time invested in learning. Ultimately, Vue.js proved to be a powerful tool; one the team would be comfortable working with again in the future. 

\subsubsection{Chart.js}
The team had initially intended to use D3.js to handle the representation of currency data through graphs, and spent a short period of time attempting to develop the application with this technology. After some difficulty with integration of D3.js in the initial weeks of development, the team had discussed the technology with fellow students who had also found D3.js difficult to incorporate into projects. Based on recommendations by fellow students, the team researched a similar technology called Chart.js.

Chart.js \cite{chartjs} allows developers to display data in a number of differently formatted graphs, all of which are capable of animation and customisation. Using the simple HTML5 \mintinline{html}{<canvas> } tag, Chart.js was much easier to adapt to than D3.js. Following brief research and some online tutorials, the team opted for the simplicity of Chart.js. Considering the graphs in this project did not need to do anything extraordinary, Chart.js was perfectly suited to this application.

\subsection{TensorFlow}
TensorFlow is an open source framework, designed to allow high performance numerical computation for machine learning purposes \cite{tensorflow}. The software can be used with a number of languages such as C and C++, with the most popular and well-documented language being Python. 

TensorFlow is based upon data flow graphs. Simply put, data flow graphs detail an input, some operations often on multiple levels, and an output. With regards to TensorFlow, each node in a graph represents a mathematical equation to be performed on the given edges which contain data (often multidimensional data arrays, described as tensors) which flow between them.

TensorFlow can be used in many applications, such as type prediction, value prediction, and image recognition; this application of TensorFlow within this project concerns value prediction. Data is loaded in, manipulated into the correct format to work with a specific model, and fed to that model. The model is first trained using historical data, in this case previous prices of Bitcoin, until it computes the correlation between actual and predicted values. Once trained, the model can be given unseen data to be tested. TensorFlow was deemed the most suitable technology for the machine learning aspect of this project, due to the the team's previous experience of the technology and its extensive documentation and tutorials available online.

\subsection{Heroku}
Having initially expected to use Microsoft Azure for application deployment, when it came time to begin thinking about deploying the application it became obvious that this technology was excessive for this instance. As mentioned previously, Microsoft Azure offers an extensive range of services, and thus can be quite complex to use even for the most basic of applications. The team had also considered Heroku in the beginning and having since gained more experience of the technology through other projects, decided it was the best option for this project. 

Heroku is an open source cloud application platform, providing extensive documentation for many facets of the deployment process and life cycle \cite{herdep}. Heroku is arguably much easier to use than Microsoft Azure, allowing a developer to create and deploy an application from scratch in a very small amount of time. Heroku also allows applications to be deployed using Git, a tool the team were very familiar with due to extensive previous experience of GitHub. It was for the ease of deployment and comprehensive, easy to follow documentation that the team ultimately decided to use Heroku when deploying this application.